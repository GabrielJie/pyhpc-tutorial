
% Default to the notebook output style

    


% Inherit from the specified cell style.




    
\documentclass{article}

    
    
    \usepackage{graphicx} % Used to insert images
    \usepackage{adjustbox} % Used to constrain images to a maximum size 
    \usepackage{color} % Allow colors to be defined
    \usepackage{enumerate} % Needed for markdown enumerations to work
    \usepackage{geometry} % Used to adjust the document margins
    \usepackage{amsmath} % Equations
    \usepackage{amssymb} % Equations
    \usepackage[mathletters]{ucs} % Extended unicode (utf-8) support
    \usepackage[utf8x]{inputenc} % Allow utf-8 characters in the tex document
    \usepackage{fancyvrb} % verbatim replacement that allows latex
    \usepackage{grffile} % extends the file name processing of package graphics 
                         % to support a larger range 
    % The hyperref package gives us a pdf with properly built
    % internal navigation ('pdf bookmarks' for the table of contents,
    % internal cross-reference links, web links for URLs, etc.)
    \usepackage{hyperref}
    \usepackage{longtable} % longtable support required by pandoc >1.10
    \usepackage{booktabs}  % table support for pandoc > 1.12.2
    

    
    
    \definecolor{orange}{cmyk}{0,0.4,0.8,0.2}
    \definecolor{darkorange}{rgb}{.71,0.21,0.01}
    \definecolor{darkgreen}{rgb}{.12,.54,.11}
    \definecolor{myteal}{rgb}{.26, .44, .56}
    \definecolor{gray}{gray}{0.45}
    \definecolor{lightgray}{gray}{.95}
    \definecolor{mediumgray}{gray}{.8}
    \definecolor{inputbackground}{rgb}{.95, .95, .85}
    \definecolor{outputbackground}{rgb}{.95, .95, .95}
    \definecolor{traceback}{rgb}{1, .95, .95}
    % ansi colors
    \definecolor{red}{rgb}{.6,0,0}
    \definecolor{green}{rgb}{0,.65,0}
    \definecolor{brown}{rgb}{0.6,0.6,0}
    \definecolor{blue}{rgb}{0,.145,.698}
    \definecolor{purple}{rgb}{.698,.145,.698}
    \definecolor{cyan}{rgb}{0,.698,.698}
    \definecolor{lightgray}{gray}{0.5}
    
    % bright ansi colors
    \definecolor{darkgray}{gray}{0.25}
    \definecolor{lightred}{rgb}{1.0,0.39,0.28}
    \definecolor{lightgreen}{rgb}{0.48,0.99,0.0}
    \definecolor{lightblue}{rgb}{0.53,0.81,0.92}
    \definecolor{lightpurple}{rgb}{0.87,0.63,0.87}
    \definecolor{lightcyan}{rgb}{0.5,1.0,0.83}
    
    % commands and environments needed by pandoc snippets
    % extracted from the output of `pandoc -s`
    \DefineVerbatimEnvironment{Highlighting}{Verbatim}{commandchars=\\\{\}}
    % Add ',fontsize=\small' for more characters per line
    \newenvironment{Shaded}{}{}
    \newcommand{\KeywordTok}[1]{\textcolor[rgb]{0.00,0.44,0.13}{\textbf{{#1}}}}
    \newcommand{\DataTypeTok}[1]{\textcolor[rgb]{0.56,0.13,0.00}{{#1}}}
    \newcommand{\DecValTok}[1]{\textcolor[rgb]{0.25,0.63,0.44}{{#1}}}
    \newcommand{\BaseNTok}[1]{\textcolor[rgb]{0.25,0.63,0.44}{{#1}}}
    \newcommand{\FloatTok}[1]{\textcolor[rgb]{0.25,0.63,0.44}{{#1}}}
    \newcommand{\CharTok}[1]{\textcolor[rgb]{0.25,0.44,0.63}{{#1}}}
    \newcommand{\StringTok}[1]{\textcolor[rgb]{0.25,0.44,0.63}{{#1}}}
    \newcommand{\CommentTok}[1]{\textcolor[rgb]{0.38,0.63,0.69}{\textit{{#1}}}}
    \newcommand{\OtherTok}[1]{\textcolor[rgb]{0.00,0.44,0.13}{{#1}}}
    \newcommand{\AlertTok}[1]{\textcolor[rgb]{1.00,0.00,0.00}{\textbf{{#1}}}}
    \newcommand{\FunctionTok}[1]{\textcolor[rgb]{0.02,0.16,0.49}{{#1}}}
    \newcommand{\RegionMarkerTok}[1]{{#1}}
    \newcommand{\ErrorTok}[1]{\textcolor[rgb]{1.00,0.00,0.00}{\textbf{{#1}}}}
    \newcommand{\NormalTok}[1]{{#1}}
    
    % Define a nice break command that doesn't care if a line doesn't already
    % exist.
    \def\br{\hspace*{\fill} \\* }
    % Math Jax compatability definitions
    \def\gt{>}
    \def\lt{<}
    % Document parameters
    \title{07\_Derived\_Fields\_in\_yt}
    
    
    

    % Pygments definitions
    
\makeatletter
\def\PY@reset{\let\PY@it=\relax \let\PY@bf=\relax%
    \let\PY@ul=\relax \let\PY@tc=\relax%
    \let\PY@bc=\relax \let\PY@ff=\relax}
\def\PY@tok#1{\csname PY@tok@#1\endcsname}
\def\PY@toks#1+{\ifx\relax#1\empty\else%
    \PY@tok{#1}\expandafter\PY@toks\fi}
\def\PY@do#1{\PY@bc{\PY@tc{\PY@ul{%
    \PY@it{\PY@bf{\PY@ff{#1}}}}}}}
\def\PY#1#2{\PY@reset\PY@toks#1+\relax+\PY@do{#2}}

\expandafter\def\csname PY@tok@gd\endcsname{\def\PY@tc##1{\textcolor[rgb]{0.63,0.00,0.00}{##1}}}
\expandafter\def\csname PY@tok@gu\endcsname{\let\PY@bf=\textbf\def\PY@tc##1{\textcolor[rgb]{0.50,0.00,0.50}{##1}}}
\expandafter\def\csname PY@tok@gt\endcsname{\def\PY@tc##1{\textcolor[rgb]{0.00,0.27,0.87}{##1}}}
\expandafter\def\csname PY@tok@gs\endcsname{\let\PY@bf=\textbf}
\expandafter\def\csname PY@tok@gr\endcsname{\def\PY@tc##1{\textcolor[rgb]{1.00,0.00,0.00}{##1}}}
\expandafter\def\csname PY@tok@cm\endcsname{\let\PY@it=\textit\def\PY@tc##1{\textcolor[rgb]{0.25,0.50,0.50}{##1}}}
\expandafter\def\csname PY@tok@vg\endcsname{\def\PY@tc##1{\textcolor[rgb]{0.10,0.09,0.49}{##1}}}
\expandafter\def\csname PY@tok@m\endcsname{\def\PY@tc##1{\textcolor[rgb]{0.40,0.40,0.40}{##1}}}
\expandafter\def\csname PY@tok@mh\endcsname{\def\PY@tc##1{\textcolor[rgb]{0.40,0.40,0.40}{##1}}}
\expandafter\def\csname PY@tok@go\endcsname{\def\PY@tc##1{\textcolor[rgb]{0.53,0.53,0.53}{##1}}}
\expandafter\def\csname PY@tok@ge\endcsname{\let\PY@it=\textit}
\expandafter\def\csname PY@tok@vc\endcsname{\def\PY@tc##1{\textcolor[rgb]{0.10,0.09,0.49}{##1}}}
\expandafter\def\csname PY@tok@il\endcsname{\def\PY@tc##1{\textcolor[rgb]{0.40,0.40,0.40}{##1}}}
\expandafter\def\csname PY@tok@cs\endcsname{\let\PY@it=\textit\def\PY@tc##1{\textcolor[rgb]{0.25,0.50,0.50}{##1}}}
\expandafter\def\csname PY@tok@cp\endcsname{\def\PY@tc##1{\textcolor[rgb]{0.74,0.48,0.00}{##1}}}
\expandafter\def\csname PY@tok@gi\endcsname{\def\PY@tc##1{\textcolor[rgb]{0.00,0.63,0.00}{##1}}}
\expandafter\def\csname PY@tok@gh\endcsname{\let\PY@bf=\textbf\def\PY@tc##1{\textcolor[rgb]{0.00,0.00,0.50}{##1}}}
\expandafter\def\csname PY@tok@ni\endcsname{\let\PY@bf=\textbf\def\PY@tc##1{\textcolor[rgb]{0.60,0.60,0.60}{##1}}}
\expandafter\def\csname PY@tok@nl\endcsname{\def\PY@tc##1{\textcolor[rgb]{0.63,0.63,0.00}{##1}}}
\expandafter\def\csname PY@tok@nn\endcsname{\let\PY@bf=\textbf\def\PY@tc##1{\textcolor[rgb]{0.00,0.00,1.00}{##1}}}
\expandafter\def\csname PY@tok@no\endcsname{\def\PY@tc##1{\textcolor[rgb]{0.53,0.00,0.00}{##1}}}
\expandafter\def\csname PY@tok@na\endcsname{\def\PY@tc##1{\textcolor[rgb]{0.49,0.56,0.16}{##1}}}
\expandafter\def\csname PY@tok@nb\endcsname{\def\PY@tc##1{\textcolor[rgb]{0.00,0.50,0.00}{##1}}}
\expandafter\def\csname PY@tok@nc\endcsname{\let\PY@bf=\textbf\def\PY@tc##1{\textcolor[rgb]{0.00,0.00,1.00}{##1}}}
\expandafter\def\csname PY@tok@nd\endcsname{\def\PY@tc##1{\textcolor[rgb]{0.67,0.13,1.00}{##1}}}
\expandafter\def\csname PY@tok@ne\endcsname{\let\PY@bf=\textbf\def\PY@tc##1{\textcolor[rgb]{0.82,0.25,0.23}{##1}}}
\expandafter\def\csname PY@tok@nf\endcsname{\def\PY@tc##1{\textcolor[rgb]{0.00,0.00,1.00}{##1}}}
\expandafter\def\csname PY@tok@si\endcsname{\let\PY@bf=\textbf\def\PY@tc##1{\textcolor[rgb]{0.73,0.40,0.53}{##1}}}
\expandafter\def\csname PY@tok@s2\endcsname{\def\PY@tc##1{\textcolor[rgb]{0.73,0.13,0.13}{##1}}}
\expandafter\def\csname PY@tok@vi\endcsname{\def\PY@tc##1{\textcolor[rgb]{0.10,0.09,0.49}{##1}}}
\expandafter\def\csname PY@tok@nt\endcsname{\let\PY@bf=\textbf\def\PY@tc##1{\textcolor[rgb]{0.00,0.50,0.00}{##1}}}
\expandafter\def\csname PY@tok@nv\endcsname{\def\PY@tc##1{\textcolor[rgb]{0.10,0.09,0.49}{##1}}}
\expandafter\def\csname PY@tok@s1\endcsname{\def\PY@tc##1{\textcolor[rgb]{0.73,0.13,0.13}{##1}}}
\expandafter\def\csname PY@tok@sh\endcsname{\def\PY@tc##1{\textcolor[rgb]{0.73,0.13,0.13}{##1}}}
\expandafter\def\csname PY@tok@sc\endcsname{\def\PY@tc##1{\textcolor[rgb]{0.73,0.13,0.13}{##1}}}
\expandafter\def\csname PY@tok@sx\endcsname{\def\PY@tc##1{\textcolor[rgb]{0.00,0.50,0.00}{##1}}}
\expandafter\def\csname PY@tok@bp\endcsname{\def\PY@tc##1{\textcolor[rgb]{0.00,0.50,0.00}{##1}}}
\expandafter\def\csname PY@tok@c1\endcsname{\let\PY@it=\textit\def\PY@tc##1{\textcolor[rgb]{0.25,0.50,0.50}{##1}}}
\expandafter\def\csname PY@tok@kc\endcsname{\let\PY@bf=\textbf\def\PY@tc##1{\textcolor[rgb]{0.00,0.50,0.00}{##1}}}
\expandafter\def\csname PY@tok@c\endcsname{\let\PY@it=\textit\def\PY@tc##1{\textcolor[rgb]{0.25,0.50,0.50}{##1}}}
\expandafter\def\csname PY@tok@mf\endcsname{\def\PY@tc##1{\textcolor[rgb]{0.40,0.40,0.40}{##1}}}
\expandafter\def\csname PY@tok@err\endcsname{\def\PY@bc##1{\setlength{\fboxsep}{0pt}\fcolorbox[rgb]{1.00,0.00,0.00}{1,1,1}{\strut ##1}}}
\expandafter\def\csname PY@tok@kd\endcsname{\let\PY@bf=\textbf\def\PY@tc##1{\textcolor[rgb]{0.00,0.50,0.00}{##1}}}
\expandafter\def\csname PY@tok@ss\endcsname{\def\PY@tc##1{\textcolor[rgb]{0.10,0.09,0.49}{##1}}}
\expandafter\def\csname PY@tok@sr\endcsname{\def\PY@tc##1{\textcolor[rgb]{0.73,0.40,0.53}{##1}}}
\expandafter\def\csname PY@tok@mo\endcsname{\def\PY@tc##1{\textcolor[rgb]{0.40,0.40,0.40}{##1}}}
\expandafter\def\csname PY@tok@kn\endcsname{\let\PY@bf=\textbf\def\PY@tc##1{\textcolor[rgb]{0.00,0.50,0.00}{##1}}}
\expandafter\def\csname PY@tok@mi\endcsname{\def\PY@tc##1{\textcolor[rgb]{0.40,0.40,0.40}{##1}}}
\expandafter\def\csname PY@tok@gp\endcsname{\let\PY@bf=\textbf\def\PY@tc##1{\textcolor[rgb]{0.00,0.00,0.50}{##1}}}
\expandafter\def\csname PY@tok@o\endcsname{\def\PY@tc##1{\textcolor[rgb]{0.40,0.40,0.40}{##1}}}
\expandafter\def\csname PY@tok@kr\endcsname{\let\PY@bf=\textbf\def\PY@tc##1{\textcolor[rgb]{0.00,0.50,0.00}{##1}}}
\expandafter\def\csname PY@tok@s\endcsname{\def\PY@tc##1{\textcolor[rgb]{0.73,0.13,0.13}{##1}}}
\expandafter\def\csname PY@tok@kp\endcsname{\def\PY@tc##1{\textcolor[rgb]{0.00,0.50,0.00}{##1}}}
\expandafter\def\csname PY@tok@w\endcsname{\def\PY@tc##1{\textcolor[rgb]{0.73,0.73,0.73}{##1}}}
\expandafter\def\csname PY@tok@kt\endcsname{\def\PY@tc##1{\textcolor[rgb]{0.69,0.00,0.25}{##1}}}
\expandafter\def\csname PY@tok@ow\endcsname{\let\PY@bf=\textbf\def\PY@tc##1{\textcolor[rgb]{0.67,0.13,1.00}{##1}}}
\expandafter\def\csname PY@tok@sb\endcsname{\def\PY@tc##1{\textcolor[rgb]{0.73,0.13,0.13}{##1}}}
\expandafter\def\csname PY@tok@k\endcsname{\let\PY@bf=\textbf\def\PY@tc##1{\textcolor[rgb]{0.00,0.50,0.00}{##1}}}
\expandafter\def\csname PY@tok@se\endcsname{\let\PY@bf=\textbf\def\PY@tc##1{\textcolor[rgb]{0.73,0.40,0.13}{##1}}}
\expandafter\def\csname PY@tok@sd\endcsname{\let\PY@it=\textit\def\PY@tc##1{\textcolor[rgb]{0.73,0.13,0.13}{##1}}}

\def\PYZbs{\char`\\}
\def\PYZus{\char`\_}
\def\PYZob{\char`\{}
\def\PYZcb{\char`\}}
\def\PYZca{\char`\^}
\def\PYZam{\char`\&}
\def\PYZlt{\char`\<}
\def\PYZgt{\char`\>}
\def\PYZsh{\char`\#}
\def\PYZpc{\char`\%}
\def\PYZdl{\char`\$}
\def\PYZhy{\char`\-}
\def\PYZsq{\char`\'}
\def\PYZdq{\char`\"}
\def\PYZti{\char`\~}
% for compatibility with earlier versions
\def\PYZat{@}
\def\PYZlb{[}
\def\PYZrb{]}
\makeatother


    % Exact colors from NB
    \definecolor{incolor}{rgb}{0.0, 0.0, 0.5}
    \definecolor{outcolor}{rgb}{0.545, 0.0, 0.0}



    
    % Prevent overflowing lines due to hard-to-break entities
    \sloppy 
    % Setup hyperref package
    \hypersetup{
      breaklinks=true,  % so long urls are correctly broken across lines
      colorlinks=true,
      urlcolor=blue,
      linkcolor=darkorange,
      citecolor=darkgreen,
      }
    % Slightly bigger margins than the latex defaults
    
    \geometry{verbose,tmargin=1in,bmargin=1in,lmargin=1in,rmargin=1in}
    
    

    \begin{document}
    
    
    \maketitle
    
    

    
    \section{Derived Fields and Profiles}\label{derived-fields-and-profiles}

One of the most powerful features in yt is the ability to create derived
fields that act and look exactly like fields that exist on disk. This
means that they will be generated on demand and can be used anywhere a
field that exists on disk would be used. Additionally, you can create
them by just writing python functions.

    \begin{Verbatim}[commandchars=\\\{\}]
{\color{incolor}In [{\color{incolor}}]:} \PY{o}{\PYZpc{}}\PY{k}{matplotlib} \PY{n}{inline}
       \PY{k+kn}{import} \PY{n+nn}{yt}
       \PY{k+kn}{import} \PY{n+nn}{numpy} \PY{k+kn}{as} \PY{n+nn}{np}
       \PY{k+kn}{from} \PY{n+nn}{yt} \PY{k+kn}{import} \PY{n}{derived\PYZus{}field}
       \PY{k+kn}{from} \PY{n+nn}{matplotlib} \PY{k+kn}{import} \PY{n}{pylab}
\end{Verbatim}

    \subsection{Derived Fields}\label{derived-fields}

This is an example of the simplest possible way to create a derived
field. All derived fields are defined by a function and some metadata;
that metadata can include units, LaTeX-friendly names, conversion
factors, and so on. Fields can be defined in the way in the next cell.
What this does is create a function which accepts two arguments and then
provide the units for that field. In this case, our field is
\texttt{dinosaurs} and our units are \texttt{K*cm/s}. The function
itself can access any fields that are in the simulation, and it does so
by requesting data from the object called \texttt{data}.

    \begin{Verbatim}[commandchars=\\\{\}]
{\color{incolor}In [{\color{incolor}}]:} \PY{n+nd}{@derived\PYZus{}field}\PY{p}{(}\PY{n}{name} \PY{o}{=} \PY{l+s}{\PYZdq{}}\PY{l+s}{dinosaurs}\PY{l+s}{\PYZdq{}}\PY{p}{,} \PY{n}{units} \PY{o}{=} \PY{l+s}{\PYZdq{}}\PY{l+s}{K * cm/s}\PY{l+s}{\PYZdq{}}\PY{p}{)}
       \PY{k}{def} \PY{n+nf}{\PYZus{}dinos}\PY{p}{(}\PY{n}{field}\PY{p}{,} \PY{n}{data}\PY{p}{)}\PY{p}{:}
           \PY{k}{return} \PY{n}{data}\PY{p}{[}\PY{l+s}{\PYZdq{}}\PY{l+s}{temperature}\PY{l+s}{\PYZdq{}}\PY{p}{]} \PY{o}{*} \PY{n}{data}\PY{p}{[}\PY{l+s}{\PYZdq{}}\PY{l+s}{velocity\PYZus{}magnitude}\PY{l+s}{\PYZdq{}}\PY{p}{]}
\end{Verbatim}

    One important thing to note is that derived fields must be defined
\emph{before} any datasets are loaded. Let's load up our data and take a
look at some quantities.

    \begin{Verbatim}[commandchars=\\\{\}]
{\color{incolor}In [{\color{incolor}}]:} \PY{n}{ds} \PY{o}{=} \PY{n}{yt}\PY{o}{.}\PY{n}{load}\PY{p}{(}\PY{l+s}{\PYZdq{}}\PY{l+s}{IsolatedGalaxy/galaxy0030/galaxy0030}\PY{l+s}{\PYZdq{}}\PY{p}{)}
       \PY{n}{dd} \PY{o}{=} \PY{n}{ds}\PY{o}{.}\PY{n}{all\PYZus{}data}\PY{p}{(}\PY{p}{)}
       \PY{k}{print} \PY{n}{dd}\PY{o}{.}\PY{n}{quantities}\PY{o}{.}\PY{n}{keys}\PY{p}{(}\PY{p}{)}
\end{Verbatim}

    One interesting question is, what are the minimum and maximum values of
dinosaur production rates in our isolated galaxy? We can do that by
examining the \texttt{extrema} quantity -- the exact same way that we
would for density, temperature, and so on.

    \begin{Verbatim}[commandchars=\\\{\}]
{\color{incolor}In [{\color{incolor}}]:} \PY{k}{print} \PY{n}{dd}\PY{o}{.}\PY{n}{quantities}\PY{o}{.}\PY{n}{extrema}\PY{p}{(}\PY{l+s}{\PYZdq{}}\PY{l+s}{dinosaurs}\PY{l+s}{\PYZdq{}}\PY{p}{)}
\end{Verbatim}

    We can do the same for the average quantities as well.

    \begin{Verbatim}[commandchars=\\\{\}]
{\color{incolor}In [{\color{incolor}}]:} \PY{k}{print} \PY{n}{dd}\PY{o}{.}\PY{n}{quantities}\PY{o}{.}\PY{n}{weighted\PYZus{}average\PYZus{}quantity}\PY{p}{(}\PY{l+s}{\PYZdq{}}\PY{l+s}{dinosaurs}\PY{l+s}{\PYZdq{}}\PY{p}{,} \PY{n}{weight}\PY{o}{=}\PY{l+s}{\PYZdq{}}\PY{l+s}{temperature}\PY{l+s}{\PYZdq{}}\PY{p}{)}
\end{Verbatim}

    \subsection{A Few Other Quantities}\label{a-few-other-quantities}

We can ask other quantities of our data, as well. For instance, this
sequence of operations will find the most dense point, center a sphere
on it, calculate the bulk velocity of that sphere, calculate the
baryonic angular momentum vector, and then the density extrema. All of
this is done in a memory conservative way: if you have an absolutely
enormous dataset, yt will split that dataset into pieces, apply
intermediate reductions and then a final reduction to calculate your
quantity.

    \begin{Verbatim}[commandchars=\\\{\}]
{\color{incolor}In [{\color{incolor}}]:} \PY{n}{sp} \PY{o}{=} \PY{n}{ds}\PY{o}{.}\PY{n}{sphere}\PY{p}{(}\PY{l+s}{\PYZdq{}}\PY{l+s}{max}\PY{l+s}{\PYZdq{}}\PY{p}{,} \PY{p}{(}\PY{l+m+mf}{10.0}\PY{p}{,} \PY{l+s}{\PYZsq{}}\PY{l+s}{kpc}\PY{l+s}{\PYZsq{}}\PY{p}{)}\PY{p}{)}
       \PY{n}{bv} \PY{o}{=} \PY{n}{sp}\PY{o}{.}\PY{n}{quantities}\PY{o}{.}\PY{n}{bulk\PYZus{}velocity}\PY{p}{(}\PY{p}{)}
       \PY{n}{L} \PY{o}{=} \PY{n}{sp}\PY{o}{.}\PY{n}{quantities}\PY{o}{.}\PY{n}{angular\PYZus{}momentum\PYZus{}vector}\PY{p}{(}\PY{p}{)}
       \PY{n}{rho\PYZus{}min}\PY{p}{,} \PY{n}{rho\PYZus{}max} \PY{o}{=} \PY{n}{sp}\PY{o}{.}\PY{n}{quantities}\PY{o}{.}\PY{n}{extrema}\PY{p}{(}\PY{l+s}{\PYZdq{}}\PY{l+s}{density}\PY{l+s}{\PYZdq{}}\PY{p}{)}
       \PY{k}{print} \PY{n}{bv}
       \PY{k}{print} \PY{n}{L}
       \PY{k}{print} \PY{n}{rho\PYZus{}min}\PY{p}{,} \PY{n}{rho\PYZus{}max}
\end{Verbatim}

    \subsection{Profiles}\label{profiles}

yt provides the ability to bin in 1, 2 and 3 dimensions. This means
discretizing in one or more dimensions of phase space (density,
temperature, etc) and then calculating either the total value of a field
in each bin or the average value of a field in each bin.

We do this using the objects \texttt{Profile1D}, \texttt{Profile2D}, and
\texttt{Profile3D}. The first two are the most common since they are the
easiest to visualize.

This first set of commands manually creates a profile object the sphere
we created earlier, binned in 32 bins according to density between
\texttt{rho\_min} and \texttt{rho\_max}, and then takes the
density-weighted average of the fields \texttt{temperature} and
(previously-defined) \texttt{dinosaurs}. We then plot it in a loglog
plot.

    \begin{Verbatim}[commandchars=\\\{\}]
{\color{incolor}In [{\color{incolor}}]:} \PY{n}{prof} \PY{o}{=} \PY{n}{yt}\PY{o}{.}\PY{n}{Profile1D}\PY{p}{(}\PY{n}{sp}\PY{p}{,} \PY{l+s}{\PYZdq{}}\PY{l+s}{density}\PY{l+s}{\PYZdq{}}\PY{p}{,} \PY{l+m+mi}{32}\PY{p}{,} \PY{n}{rho\PYZus{}min}\PY{p}{,} \PY{n}{rho\PYZus{}max}\PY{p}{,} \PY{n+nb+bp}{True}\PY{p}{,} \PY{n}{weight\PYZus{}field}\PY{o}{=}\PY{l+s}{\PYZdq{}}\PY{l+s}{cell\PYZus{}mass}\PY{l+s}{\PYZdq{}}\PY{p}{)}
       \PY{n}{prof}\PY{o}{.}\PY{n}{add\PYZus{}fields}\PY{p}{(}\PY{p}{[}\PY{l+s}{\PYZdq{}}\PY{l+s}{temperature}\PY{l+s}{\PYZdq{}}\PY{p}{,}\PY{l+s}{\PYZdq{}}\PY{l+s}{dinosaurs}\PY{l+s}{\PYZdq{}}\PY{p}{]}\PY{p}{)}
       \PY{n}{pylab}\PY{o}{.}\PY{n}{loglog}\PY{p}{(}\PY{n}{np}\PY{o}{.}\PY{n}{array}\PY{p}{(}\PY{n}{prof}\PY{o}{.}\PY{n}{x}\PY{p}{)}\PY{p}{,} \PY{n}{np}\PY{o}{.}\PY{n}{array}\PY{p}{(}\PY{n}{prof}\PY{p}{[}\PY{l+s}{\PYZdq{}}\PY{l+s}{temperature}\PY{l+s}{\PYZdq{}}\PY{p}{]}\PY{p}{)}\PY{p}{,} \PY{l+s}{\PYZdq{}}\PY{l+s}{\PYZhy{}x}\PY{l+s}{\PYZdq{}}\PY{p}{)}
       \PY{n}{pylab}\PY{o}{.}\PY{n}{xlabel}\PY{p}{(}\PY{l+s}{\PYZsq{}}\PY{l+s}{Density \PYZdl{}(g/cm\PYZca{}3)\PYZdl{}}\PY{l+s}{\PYZsq{}}\PY{p}{)}
       \PY{n}{pylab}\PY{o}{.}\PY{n}{ylabel}\PY{p}{(}\PY{l+s}{\PYZsq{}}\PY{l+s}{Temperature \PYZdl{}(K)\PYZdl{}}\PY{l+s}{\PYZsq{}}\PY{p}{)}
\end{Verbatim}

    Now we plot the \texttt{dinosaurs} field.

    \begin{Verbatim}[commandchars=\\\{\}]
{\color{incolor}In [{\color{incolor}}]:} \PY{n}{pylab}\PY{o}{.}\PY{n}{loglog}\PY{p}{(}\PY{n}{np}\PY{o}{.}\PY{n}{array}\PY{p}{(}\PY{n}{prof}\PY{o}{.}\PY{n}{x}\PY{p}{)}\PY{p}{,} \PY{n}{np}\PY{o}{.}\PY{n}{array}\PY{p}{(}\PY{n}{prof}\PY{p}{[}\PY{l+s}{\PYZdq{}}\PY{l+s}{dinosaurs}\PY{l+s}{\PYZdq{}}\PY{p}{]}\PY{p}{)}\PY{p}{,} \PY{l+s}{\PYZsq{}}\PY{l+s}{\PYZhy{}x}\PY{l+s}{\PYZsq{}}\PY{p}{)}
       \PY{n}{pylab}\PY{o}{.}\PY{n}{xlabel}\PY{p}{(}\PY{l+s}{\PYZsq{}}\PY{l+s}{Density \PYZdl{}(g/cm\PYZca{}3)\PYZdl{}}\PY{l+s}{\PYZsq{}}\PY{p}{)}
       \PY{n}{pylab}\PY{o}{.}\PY{n}{ylabel}\PY{p}{(}\PY{l+s}{\PYZsq{}}\PY{l+s}{Dinosaurs \PYZdl{}(K cm / s)\PYZdl{}}\PY{l+s}{\PYZsq{}}\PY{p}{)}
\end{Verbatim}

    If we want to see the total mass in every bin, we profile the
\texttt{cell\_mass} field with no weight. Specifying
\texttt{weight=None} will simply take the total value in every bin and
add that up.

    \begin{Verbatim}[commandchars=\\\{\}]
{\color{incolor}In [{\color{incolor}}]:} \PY{n}{prof} \PY{o}{=} \PY{n}{yt}\PY{o}{.}\PY{n}{Profile1D}\PY{p}{(}\PY{n}{sp}\PY{p}{,} \PY{l+s}{\PYZdq{}}\PY{l+s}{density}\PY{l+s}{\PYZdq{}}\PY{p}{,} \PY{l+m+mi}{32}\PY{p}{,} \PY{n}{rho\PYZus{}min}\PY{p}{,} \PY{n}{rho\PYZus{}max}\PY{p}{,} \PY{n+nb+bp}{True}\PY{p}{,} \PY{n}{weight\PYZus{}field}\PY{o}{=}\PY{n+nb+bp}{None}\PY{p}{)}
       \PY{n}{prof}\PY{o}{.}\PY{n}{add\PYZus{}fields}\PY{p}{(}\PY{p}{[}\PY{l+s}{\PYZdq{}}\PY{l+s}{cell\PYZus{}mass}\PY{l+s}{\PYZdq{}}\PY{p}{]}\PY{p}{)}
       \PY{n}{pylab}\PY{o}{.}\PY{n}{loglog}\PY{p}{(}\PY{n}{np}\PY{o}{.}\PY{n}{array}\PY{p}{(}\PY{n}{prof}\PY{o}{.}\PY{n}{x}\PY{p}{)}\PY{p}{,} \PY{n}{np}\PY{o}{.}\PY{n}{array}\PY{p}{(}\PY{n}{prof}\PY{p}{[}\PY{l+s}{\PYZdq{}}\PY{l+s}{cell\PYZus{}mass}\PY{l+s}{\PYZdq{}}\PY{p}{]}\PY{o}{.}\PY{n}{in\PYZus{}units}\PY{p}{(}\PY{l+s}{\PYZdq{}}\PY{l+s}{Msun}\PY{l+s}{\PYZdq{}}\PY{p}{)}\PY{p}{)}\PY{p}{,} \PY{l+s}{\PYZsq{}}\PY{l+s}{\PYZhy{}x}\PY{l+s}{\PYZsq{}}\PY{p}{)}
       \PY{n}{pylab}\PY{o}{.}\PY{n}{xlabel}\PY{p}{(}\PY{l+s}{\PYZsq{}}\PY{l+s}{Density \PYZdl{}(g/cm\PYZca{}3)\PYZdl{}}\PY{l+s}{\PYZsq{}}\PY{p}{)}
       \PY{n}{pylab}\PY{o}{.}\PY{n}{ylabel}\PY{p}{(}\PY{l+s}{\PYZsq{}}\PY{l+s}{Cell mass \PYZdl{}(M\PYZus{}}\PY{l+s}{\PYZbs{}}\PY{l+s}{odot)\PYZdl{}}\PY{l+s}{\PYZsq{}}\PY{p}{)}
\end{Verbatim}

    In addition to the low-level \texttt{ProfileND} interface, it's also
quite straightforward to quickly create plots of profiles using the
\texttt{ProfilePlot} class. Let's redo the last plot using
\texttt{ProfilePlot}

    \begin{Verbatim}[commandchars=\\\{\}]
{\color{incolor}In [{\color{incolor}}]:} \PY{n}{prof} \PY{o}{=} \PY{n}{yt}\PY{o}{.}\PY{n}{ProfilePlot}\PY{p}{(}\PY{n}{sp}\PY{p}{,} \PY{l+s}{\PYZsq{}}\PY{l+s}{density}\PY{l+s}{\PYZsq{}}\PY{p}{,} \PY{l+s}{\PYZsq{}}\PY{l+s}{cell\PYZus{}mass}\PY{l+s}{\PYZsq{}}\PY{p}{,} \PY{n}{weight\PYZus{}field}\PY{o}{=}\PY{n+nb+bp}{None}\PY{p}{)}
       \PY{n}{prof}\PY{o}{.}\PY{n}{set\PYZus{}unit}\PY{p}{(}\PY{l+s}{\PYZsq{}}\PY{l+s}{cell\PYZus{}mass}\PY{l+s}{\PYZsq{}}\PY{p}{,} \PY{l+s}{\PYZsq{}}\PY{l+s}{Msun}\PY{l+s}{\PYZsq{}}\PY{p}{)}
       \PY{n}{prof}\PY{o}{.}\PY{n}{show}\PY{p}{(}\PY{p}{)}
\end{Verbatim}

    \subsection{Field Parameters}\label{field-parameters}

Field parameters are a method of passing information to derived fields.
For instance, you might pass in information about a vector you want to
use as a basis for a coordinate transformation. yt often uses things
like \texttt{bulk\_velocity} to identify velocities that should be
subtracted off. Here we show how that works:

    \begin{Verbatim}[commandchars=\\\{\}]
{\color{incolor}In [{\color{incolor}}]:} \PY{n}{sp\PYZus{}small} \PY{o}{=} \PY{n}{ds}\PY{o}{.}\PY{n}{sphere}\PY{p}{(}\PY{l+s}{\PYZdq{}}\PY{l+s}{max}\PY{l+s}{\PYZdq{}}\PY{p}{,} \PY{p}{(}\PY{l+m+mf}{50.0}\PY{p}{,} \PY{l+s}{\PYZsq{}}\PY{l+s}{kpc}\PY{l+s}{\PYZsq{}}\PY{p}{)}\PY{p}{)}
       \PY{n}{bv} \PY{o}{=} \PY{n}{sp\PYZus{}small}\PY{o}{.}\PY{n}{quantities}\PY{o}{.}\PY{n}{bulk\PYZus{}velocity}\PY{p}{(}\PY{p}{)}
       
       \PY{n}{sp} \PY{o}{=} \PY{n}{ds}\PY{o}{.}\PY{n}{sphere}\PY{p}{(}\PY{l+s}{\PYZdq{}}\PY{l+s}{max}\PY{l+s}{\PYZdq{}}\PY{p}{,} \PY{p}{(}\PY{l+m+mf}{0.1}\PY{p}{,} \PY{l+s}{\PYZsq{}}\PY{l+s}{Mpc}\PY{l+s}{\PYZsq{}}\PY{p}{)}\PY{p}{)}
       \PY{n}{rv1} \PY{o}{=} \PY{n}{sp}\PY{o}{.}\PY{n}{quantities}\PY{o}{.}\PY{n}{extrema}\PY{p}{(}\PY{l+s}{\PYZdq{}}\PY{l+s}{radial\PYZus{}velocity}\PY{l+s}{\PYZdq{}}\PY{p}{)}
       
       \PY{n}{sp}\PY{o}{.}\PY{n}{clear\PYZus{}data}\PY{p}{(}\PY{p}{)}
       \PY{n}{sp}\PY{o}{.}\PY{n}{set\PYZus{}field\PYZus{}parameter}\PY{p}{(}\PY{l+s}{\PYZdq{}}\PY{l+s}{bulk\PYZus{}velocity}\PY{l+s}{\PYZdq{}}\PY{p}{,} \PY{n}{bv}\PY{p}{)}
       \PY{n}{rv2} \PY{o}{=} \PY{n}{sp}\PY{o}{.}\PY{n}{quantities}\PY{o}{.}\PY{n}{extrema}\PY{p}{(}\PY{l+s}{\PYZdq{}}\PY{l+s}{radial\PYZus{}velocity}\PY{l+s}{\PYZdq{}}\PY{p}{)}
       
       \PY{k}{print} \PY{n}{bv}
       \PY{k}{print} \PY{n}{rv1}
       \PY{k}{print} \PY{n}{rv2}
\end{Verbatim}


    % Add a bibliography block to the postdoc
    
    
    
    \end{document}
