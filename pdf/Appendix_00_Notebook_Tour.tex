
% Default to the notebook output style

    


% Inherit from the specified cell style.




    
\documentclass{article}

    
    
    \usepackage{graphicx} % Used to insert images
    \usepackage{adjustbox} % Used to constrain images to a maximum size 
    \usepackage{color} % Allow colors to be defined
    \usepackage{enumerate} % Needed for markdown enumerations to work
    \usepackage{geometry} % Used to adjust the document margins
    \usepackage{amsmath} % Equations
    \usepackage{amssymb} % Equations
    \usepackage[mathletters]{ucs} % Extended unicode (utf-8) support
    \usepackage[utf8x]{inputenc} % Allow utf-8 characters in the tex document
    \usepackage{fancyvrb} % verbatim replacement that allows latex
    \usepackage{grffile} % extends the file name processing of package graphics 
                         % to support a larger range 
    % The hyperref package gives us a pdf with properly built
    % internal navigation ('pdf bookmarks' for the table of contents,
    % internal cross-reference links, web links for URLs, etc.)
    \usepackage{hyperref}
    \usepackage{longtable} % longtable support required by pandoc >1.10
    \usepackage{booktabs}  % table support for pandoc > 1.12.2
    

    
    
    \definecolor{orange}{cmyk}{0,0.4,0.8,0.2}
    \definecolor{darkorange}{rgb}{.71,0.21,0.01}
    \definecolor{darkgreen}{rgb}{.12,.54,.11}
    \definecolor{myteal}{rgb}{.26, .44, .56}
    \definecolor{gray}{gray}{0.45}
    \definecolor{lightgray}{gray}{.95}
    \definecolor{mediumgray}{gray}{.8}
    \definecolor{inputbackground}{rgb}{.95, .95, .85}
    \definecolor{outputbackground}{rgb}{.95, .95, .95}
    \definecolor{traceback}{rgb}{1, .95, .95}
    % ansi colors
    \definecolor{red}{rgb}{.6,0,0}
    \definecolor{green}{rgb}{0,.65,0}
    \definecolor{brown}{rgb}{0.6,0.6,0}
    \definecolor{blue}{rgb}{0,.145,.698}
    \definecolor{purple}{rgb}{.698,.145,.698}
    \definecolor{cyan}{rgb}{0,.698,.698}
    \definecolor{lightgray}{gray}{0.5}
    
    % bright ansi colors
    \definecolor{darkgray}{gray}{0.25}
    \definecolor{lightred}{rgb}{1.0,0.39,0.28}
    \definecolor{lightgreen}{rgb}{0.48,0.99,0.0}
    \definecolor{lightblue}{rgb}{0.53,0.81,0.92}
    \definecolor{lightpurple}{rgb}{0.87,0.63,0.87}
    \definecolor{lightcyan}{rgb}{0.5,1.0,0.83}
    
    % commands and environments needed by pandoc snippets
    % extracted from the output of `pandoc -s`
    \DefineVerbatimEnvironment{Highlighting}{Verbatim}{commandchars=\\\{\}}
    % Add ',fontsize=\small' for more characters per line
    \newenvironment{Shaded}{}{}
    \newcommand{\KeywordTok}[1]{\textcolor[rgb]{0.00,0.44,0.13}{\textbf{{#1}}}}
    \newcommand{\DataTypeTok}[1]{\textcolor[rgb]{0.56,0.13,0.00}{{#1}}}
    \newcommand{\DecValTok}[1]{\textcolor[rgb]{0.25,0.63,0.44}{{#1}}}
    \newcommand{\BaseNTok}[1]{\textcolor[rgb]{0.25,0.63,0.44}{{#1}}}
    \newcommand{\FloatTok}[1]{\textcolor[rgb]{0.25,0.63,0.44}{{#1}}}
    \newcommand{\CharTok}[1]{\textcolor[rgb]{0.25,0.44,0.63}{{#1}}}
    \newcommand{\StringTok}[1]{\textcolor[rgb]{0.25,0.44,0.63}{{#1}}}
    \newcommand{\CommentTok}[1]{\textcolor[rgb]{0.38,0.63,0.69}{\textit{{#1}}}}
    \newcommand{\OtherTok}[1]{\textcolor[rgb]{0.00,0.44,0.13}{{#1}}}
    \newcommand{\AlertTok}[1]{\textcolor[rgb]{1.00,0.00,0.00}{\textbf{{#1}}}}
    \newcommand{\FunctionTok}[1]{\textcolor[rgb]{0.02,0.16,0.49}{{#1}}}
    \newcommand{\RegionMarkerTok}[1]{{#1}}
    \newcommand{\ErrorTok}[1]{\textcolor[rgb]{1.00,0.00,0.00}{\textbf{{#1}}}}
    \newcommand{\NormalTok}[1]{{#1}}
    
    % Define a nice break command that doesn't care if a line doesn't already
    % exist.
    \def\br{\hspace*{\fill} \\* }
    % Math Jax compatability definitions
    \def\gt{>}
    \def\lt{<}
    % Document parameters
    \title{Appendix\_00\_Notebook\_Tour}
    
    
    

    % Pygments definitions
    
\makeatletter
\def\PY@reset{\let\PY@it=\relax \let\PY@bf=\relax%
    \let\PY@ul=\relax \let\PY@tc=\relax%
    \let\PY@bc=\relax \let\PY@ff=\relax}
\def\PY@tok#1{\csname PY@tok@#1\endcsname}
\def\PY@toks#1+{\ifx\relax#1\empty\else%
    \PY@tok{#1}\expandafter\PY@toks\fi}
\def\PY@do#1{\PY@bc{\PY@tc{\PY@ul{%
    \PY@it{\PY@bf{\PY@ff{#1}}}}}}}
\def\PY#1#2{\PY@reset\PY@toks#1+\relax+\PY@do{#2}}

\expandafter\def\csname PY@tok@gd\endcsname{\def\PY@tc##1{\textcolor[rgb]{0.63,0.00,0.00}{##1}}}
\expandafter\def\csname PY@tok@gu\endcsname{\let\PY@bf=\textbf\def\PY@tc##1{\textcolor[rgb]{0.50,0.00,0.50}{##1}}}
\expandafter\def\csname PY@tok@gt\endcsname{\def\PY@tc##1{\textcolor[rgb]{0.00,0.27,0.87}{##1}}}
\expandafter\def\csname PY@tok@gs\endcsname{\let\PY@bf=\textbf}
\expandafter\def\csname PY@tok@gr\endcsname{\def\PY@tc##1{\textcolor[rgb]{1.00,0.00,0.00}{##1}}}
\expandafter\def\csname PY@tok@cm\endcsname{\let\PY@it=\textit\def\PY@tc##1{\textcolor[rgb]{0.25,0.50,0.50}{##1}}}
\expandafter\def\csname PY@tok@vg\endcsname{\def\PY@tc##1{\textcolor[rgb]{0.10,0.09,0.49}{##1}}}
\expandafter\def\csname PY@tok@m\endcsname{\def\PY@tc##1{\textcolor[rgb]{0.40,0.40,0.40}{##1}}}
\expandafter\def\csname PY@tok@mh\endcsname{\def\PY@tc##1{\textcolor[rgb]{0.40,0.40,0.40}{##1}}}
\expandafter\def\csname PY@tok@go\endcsname{\def\PY@tc##1{\textcolor[rgb]{0.53,0.53,0.53}{##1}}}
\expandafter\def\csname PY@tok@ge\endcsname{\let\PY@it=\textit}
\expandafter\def\csname PY@tok@vc\endcsname{\def\PY@tc##1{\textcolor[rgb]{0.10,0.09,0.49}{##1}}}
\expandafter\def\csname PY@tok@il\endcsname{\def\PY@tc##1{\textcolor[rgb]{0.40,0.40,0.40}{##1}}}
\expandafter\def\csname PY@tok@cs\endcsname{\let\PY@it=\textit\def\PY@tc##1{\textcolor[rgb]{0.25,0.50,0.50}{##1}}}
\expandafter\def\csname PY@tok@cp\endcsname{\def\PY@tc##1{\textcolor[rgb]{0.74,0.48,0.00}{##1}}}
\expandafter\def\csname PY@tok@gi\endcsname{\def\PY@tc##1{\textcolor[rgb]{0.00,0.63,0.00}{##1}}}
\expandafter\def\csname PY@tok@gh\endcsname{\let\PY@bf=\textbf\def\PY@tc##1{\textcolor[rgb]{0.00,0.00,0.50}{##1}}}
\expandafter\def\csname PY@tok@ni\endcsname{\let\PY@bf=\textbf\def\PY@tc##1{\textcolor[rgb]{0.60,0.60,0.60}{##1}}}
\expandafter\def\csname PY@tok@nl\endcsname{\def\PY@tc##1{\textcolor[rgb]{0.63,0.63,0.00}{##1}}}
\expandafter\def\csname PY@tok@nn\endcsname{\let\PY@bf=\textbf\def\PY@tc##1{\textcolor[rgb]{0.00,0.00,1.00}{##1}}}
\expandafter\def\csname PY@tok@no\endcsname{\def\PY@tc##1{\textcolor[rgb]{0.53,0.00,0.00}{##1}}}
\expandafter\def\csname PY@tok@na\endcsname{\def\PY@tc##1{\textcolor[rgb]{0.49,0.56,0.16}{##1}}}
\expandafter\def\csname PY@tok@nb\endcsname{\def\PY@tc##1{\textcolor[rgb]{0.00,0.50,0.00}{##1}}}
\expandafter\def\csname PY@tok@nc\endcsname{\let\PY@bf=\textbf\def\PY@tc##1{\textcolor[rgb]{0.00,0.00,1.00}{##1}}}
\expandafter\def\csname PY@tok@nd\endcsname{\def\PY@tc##1{\textcolor[rgb]{0.67,0.13,1.00}{##1}}}
\expandafter\def\csname PY@tok@ne\endcsname{\let\PY@bf=\textbf\def\PY@tc##1{\textcolor[rgb]{0.82,0.25,0.23}{##1}}}
\expandafter\def\csname PY@tok@nf\endcsname{\def\PY@tc##1{\textcolor[rgb]{0.00,0.00,1.00}{##1}}}
\expandafter\def\csname PY@tok@si\endcsname{\let\PY@bf=\textbf\def\PY@tc##1{\textcolor[rgb]{0.73,0.40,0.53}{##1}}}
\expandafter\def\csname PY@tok@s2\endcsname{\def\PY@tc##1{\textcolor[rgb]{0.73,0.13,0.13}{##1}}}
\expandafter\def\csname PY@tok@vi\endcsname{\def\PY@tc##1{\textcolor[rgb]{0.10,0.09,0.49}{##1}}}
\expandafter\def\csname PY@tok@nt\endcsname{\let\PY@bf=\textbf\def\PY@tc##1{\textcolor[rgb]{0.00,0.50,0.00}{##1}}}
\expandafter\def\csname PY@tok@nv\endcsname{\def\PY@tc##1{\textcolor[rgb]{0.10,0.09,0.49}{##1}}}
\expandafter\def\csname PY@tok@s1\endcsname{\def\PY@tc##1{\textcolor[rgb]{0.73,0.13,0.13}{##1}}}
\expandafter\def\csname PY@tok@sh\endcsname{\def\PY@tc##1{\textcolor[rgb]{0.73,0.13,0.13}{##1}}}
\expandafter\def\csname PY@tok@sc\endcsname{\def\PY@tc##1{\textcolor[rgb]{0.73,0.13,0.13}{##1}}}
\expandafter\def\csname PY@tok@sx\endcsname{\def\PY@tc##1{\textcolor[rgb]{0.00,0.50,0.00}{##1}}}
\expandafter\def\csname PY@tok@bp\endcsname{\def\PY@tc##1{\textcolor[rgb]{0.00,0.50,0.00}{##1}}}
\expandafter\def\csname PY@tok@c1\endcsname{\let\PY@it=\textit\def\PY@tc##1{\textcolor[rgb]{0.25,0.50,0.50}{##1}}}
\expandafter\def\csname PY@tok@kc\endcsname{\let\PY@bf=\textbf\def\PY@tc##1{\textcolor[rgb]{0.00,0.50,0.00}{##1}}}
\expandafter\def\csname PY@tok@c\endcsname{\let\PY@it=\textit\def\PY@tc##1{\textcolor[rgb]{0.25,0.50,0.50}{##1}}}
\expandafter\def\csname PY@tok@mf\endcsname{\def\PY@tc##1{\textcolor[rgb]{0.40,0.40,0.40}{##1}}}
\expandafter\def\csname PY@tok@err\endcsname{\def\PY@bc##1{\setlength{\fboxsep}{0pt}\fcolorbox[rgb]{1.00,0.00,0.00}{1,1,1}{\strut ##1}}}
\expandafter\def\csname PY@tok@kd\endcsname{\let\PY@bf=\textbf\def\PY@tc##1{\textcolor[rgb]{0.00,0.50,0.00}{##1}}}
\expandafter\def\csname PY@tok@ss\endcsname{\def\PY@tc##1{\textcolor[rgb]{0.10,0.09,0.49}{##1}}}
\expandafter\def\csname PY@tok@sr\endcsname{\def\PY@tc##1{\textcolor[rgb]{0.73,0.40,0.53}{##1}}}
\expandafter\def\csname PY@tok@mo\endcsname{\def\PY@tc##1{\textcolor[rgb]{0.40,0.40,0.40}{##1}}}
\expandafter\def\csname PY@tok@kn\endcsname{\let\PY@bf=\textbf\def\PY@tc##1{\textcolor[rgb]{0.00,0.50,0.00}{##1}}}
\expandafter\def\csname PY@tok@mi\endcsname{\def\PY@tc##1{\textcolor[rgb]{0.40,0.40,0.40}{##1}}}
\expandafter\def\csname PY@tok@gp\endcsname{\let\PY@bf=\textbf\def\PY@tc##1{\textcolor[rgb]{0.00,0.00,0.50}{##1}}}
\expandafter\def\csname PY@tok@o\endcsname{\def\PY@tc##1{\textcolor[rgb]{0.40,0.40,0.40}{##1}}}
\expandafter\def\csname PY@tok@kr\endcsname{\let\PY@bf=\textbf\def\PY@tc##1{\textcolor[rgb]{0.00,0.50,0.00}{##1}}}
\expandafter\def\csname PY@tok@s\endcsname{\def\PY@tc##1{\textcolor[rgb]{0.73,0.13,0.13}{##1}}}
\expandafter\def\csname PY@tok@kp\endcsname{\def\PY@tc##1{\textcolor[rgb]{0.00,0.50,0.00}{##1}}}
\expandafter\def\csname PY@tok@w\endcsname{\def\PY@tc##1{\textcolor[rgb]{0.73,0.73,0.73}{##1}}}
\expandafter\def\csname PY@tok@kt\endcsname{\def\PY@tc##1{\textcolor[rgb]{0.69,0.00,0.25}{##1}}}
\expandafter\def\csname PY@tok@ow\endcsname{\let\PY@bf=\textbf\def\PY@tc##1{\textcolor[rgb]{0.67,0.13,1.00}{##1}}}
\expandafter\def\csname PY@tok@sb\endcsname{\def\PY@tc##1{\textcolor[rgb]{0.73,0.13,0.13}{##1}}}
\expandafter\def\csname PY@tok@k\endcsname{\let\PY@bf=\textbf\def\PY@tc##1{\textcolor[rgb]{0.00,0.50,0.00}{##1}}}
\expandafter\def\csname PY@tok@se\endcsname{\let\PY@bf=\textbf\def\PY@tc##1{\textcolor[rgb]{0.73,0.40,0.13}{##1}}}
\expandafter\def\csname PY@tok@sd\endcsname{\let\PY@it=\textit\def\PY@tc##1{\textcolor[rgb]{0.73,0.13,0.13}{##1}}}

\def\PYZbs{\char`\\}
\def\PYZus{\char`\_}
\def\PYZob{\char`\{}
\def\PYZcb{\char`\}}
\def\PYZca{\char`\^}
\def\PYZam{\char`\&}
\def\PYZlt{\char`\<}
\def\PYZgt{\char`\>}
\def\PYZsh{\char`\#}
\def\PYZpc{\char`\%}
\def\PYZdl{\char`\$}
\def\PYZhy{\char`\-}
\def\PYZsq{\char`\'}
\def\PYZdq{\char`\"}
\def\PYZti{\char`\~}
% for compatibility with earlier versions
\def\PYZat{@}
\def\PYZlb{[}
\def\PYZrb{]}
\makeatother


    % Exact colors from NB
    \definecolor{incolor}{rgb}{0.0, 0.0, 0.5}
    \definecolor{outcolor}{rgb}{0.545, 0.0, 0.0}



    
    % Prevent overflowing lines due to hard-to-break entities
    \sloppy 
    % Setup hyperref package
    \hypersetup{
      breaklinks=true,  % so long urls are correctly broken across lines
      colorlinks=true,
      urlcolor=blue,
      linkcolor=darkorange,
      citecolor=darkgreen,
      }
    % Slightly bigger margins than the latex defaults
    
    \geometry{verbose,tmargin=1in,bmargin=1in,lmargin=1in,rmargin=1in}
    
    

    \begin{document}
    
    
    \maketitle
    
    

    
    This material is included with permission from IPython documentation
available online from: https://github.com/ipython/ipython-in-depth

    \section{A brief tour of the IPython
notebook}\label{a-brief-tour-of-the-ipython-notebook}

This document will give you a brief tour of the capabilities of the
IPython notebook.\\You can view its contents by scrolling around, or
execute each cell by typing \texttt{Shift-Enter}. After you conclude
this brief high-level tour, you should read the accompanying notebook
titled \texttt{01\_notebook\_introduction}, which takes a more
step-by-step approach to the features of the system.

The rest of the notebooks in this directory illustrate various other
aspects and capabilities of the IPython notebook; some of them may
require additional libraries to be executed.

\textbf{NOTE:} This notebook \emph{must} be run from its own directory,
so you must \texttt{cd} to this directory and then start the notebook,
but do \emph{not} use the \texttt{-\/-notebook-dir} option to run it
from another location.

The first thing you need to know is that you are still controlling the
same old IPython you're used to, so things like shell aliases and magic
commands still work:

    \begin{Verbatim}[commandchars=\\\{\}]
{\color{incolor}In [{\color{incolor}1}]:} \PY{n}{pwd}
\end{Verbatim}

            \begin{Verbatim}[commandchars=\\\{\}]
{\color{outcolor}Out[{\color{outcolor}1}]:} u'/Users/minrk/dev/ip/mine/docs/examples/notebooks'
\end{Verbatim}
        
    \begin{Verbatim}[commandchars=\\\{\}]
{\color{incolor}In [{\color{incolor}2}]:} \PY{n}{ls}
\end{Verbatim}

    \begin{Verbatim}[commandchars=\\\{\}]
00\_notebook\_tour.ipynb          callbacks.ipynb                 python-logo.svg
01\_notebook\_introduction.ipynb  cython\_extension.ipynb          rmagic\_extension.ipynb
Animations\_and\_Progress.ipynb   display\_protocol.ipynb          sympy.ipynb
Capturing Output.ipynb          formatting.ipynb                sympy\_quantum\_computing.ipynb
Script Magics.ipynb             octavemagic\_extension.ipynb     trapezoid\_rule.ipynb
animation.m4v                   progbar.ipynb
    \end{Verbatim}

    \begin{Verbatim}[commandchars=\\\{\}]
{\color{incolor}In [{\color{incolor}3}]:} \PY{n}{message} \PY{o}{=} \PY{l+s}{\PYZsq{}}\PY{l+s}{The IPython notebook is great!}\PY{l+s}{\PYZsq{}}
        \PY{c}{\PYZsh{} note: the echo command does not run on Windows, it\PYZsq{}s a unix command.}
        \PY{o}{!}\PY{n+nb}{echo} \PY{n+nv}{\PYZdl{}message}
\end{Verbatim}

    \begin{Verbatim}[commandchars=\\\{\}]
The IPython notebook is great!
    \end{Verbatim}


    \subsection{Plots with matplotlib}


    IPython adds an `inline' matplotlib backend, which embeds any matplotlib
figures into the notebook.

    \begin{Verbatim}[commandchars=\\\{\}]
{\color{incolor}In [{\color{incolor}4}]:} \PY{o}{\PYZpc{}}\PY{k}{pylab} \PY{n}{inline}
\end{Verbatim}

    \begin{Verbatim}[commandchars=\\\{\}]
Welcome to pylab, a matplotlib-based Python environment [backend: module://IPython.zmq.pylab.backend\_inline].
For more information, type 'help(pylab)'.
    \end{Verbatim}

    \begin{Verbatim}[commandchars=\\\{\}]
{\color{incolor}In [{\color{incolor}5}]:} \PY{n}{x} \PY{o}{=} \PY{n}{linspace}\PY{p}{(}\PY{l+m+mi}{0}\PY{p}{,} \PY{l+m+mi}{3}\PY{o}{*}\PY{n}{pi}\PY{p}{,} \PY{l+m+mi}{500}\PY{p}{)}
        \PY{n}{plot}\PY{p}{(}\PY{n}{x}\PY{p}{,} \PY{n}{sin}\PY{p}{(}\PY{n}{x}\PY{o}{*}\PY{o}{*}\PY{l+m+mi}{2}\PY{p}{)}\PY{p}{)}
        \PY{n}{title}\PY{p}{(}\PY{l+s}{\PYZsq{}}\PY{l+s}{A simple chirp}\PY{l+s}{\PYZsq{}}\PY{p}{)}\PY{p}{;}
\end{Verbatim}

    \begin{center}
    \adjustimage{max size={0.9\linewidth}{0.9\paperheight}}{Appendix_00_Notebook_Tour_files/Appendix_00_Notebook_Tour_8_0.png}
    \end{center}
    { \hspace*{\fill} \\}
    
    You can paste blocks of input with prompt markers, such as those from
\href{http://docs.python.org/tutorial/interpreter.html\#interactive-mode}{the
official Python tutorial}

    \begin{Verbatim}[commandchars=\\\{\}]
{\color{incolor}In [{\color{incolor}6}]:} \PY{o}{\PYZgt{}\PYZgt{}}\PY{o}{\PYZgt{}} \PY{n}{the\PYZus{}world\PYZus{}is\PYZus{}flat} \PY{o}{=} \PY{l+m+mi}{1}
        \PY{o}{\PYZgt{}\PYZgt{}}\PY{o}{\PYZgt{}} \PY{k}{if} \PY{n}{the\PYZus{}world\PYZus{}is\PYZus{}flat}\PY{p}{:}
        \PY{o}{.}\PY{o}{.}\PY{o}{.}     \PY{k}{print} \PY{l+s}{\PYZdq{}}\PY{l+s}{Be careful not to fall off!}\PY{l+s}{\PYZdq{}}
\end{Verbatim}

    \begin{Verbatim}[commandchars=\\\{\}]
Be careful not to fall off!
    \end{Verbatim}

    Errors are shown in informative ways:

    \begin{Verbatim}[commandchars=\\\{\}]
{\color{incolor}In [{\color{incolor}7}]:} \PY{o}{\PYZpc{}}\PY{k}{run} \PY{n}{non\PYZus{}existent\PYZus{}file}
\end{Verbatim}

    \begin{Verbatim}[commandchars=\\\{\}]
ERROR: File `u'non\_existent\_file.py'` not found.
    \end{Verbatim}

    \begin{Verbatim}[commandchars=\\\{\}]
{\color{incolor}In [{\color{incolor}8}]:} \PY{n}{x} \PY{o}{=} \PY{l+m+mi}{1}
        \PY{n}{y} \PY{o}{=} \PY{l+m+mi}{4}
        \PY{n}{z} \PY{o}{=} \PY{n}{y}\PY{o}{/}\PY{p}{(}\PY{l+m+mi}{1}\PY{o}{\PYZhy{}}\PY{n}{x}\PY{p}{)}
\end{Verbatim}

    \begin{Verbatim}[commandchars=\\\{\}]

        ---------------------------------------------------------------------------
    ZeroDivisionError                         Traceback (most recent call last)

        <ipython-input-8-dc39888fd1d2> in <module>()
          1 x = 1
          2 y = 4
    ----> 3 z = y/(1-x)
    

        ZeroDivisionError: integer division or modulo by zero

    \end{Verbatim}

    When IPython needs to display additional information (such as providing
details on an object via \texttt{x?} it will automatically invoke a
pager at the bottom of the screen:

    \begin{Verbatim}[commandchars=\\\{\}]
{\color{incolor}In [{\color{incolor}18}]:} \PY{n}{magic}
\end{Verbatim}

    \subsection{Non-blocking output of
kernel}\label{non-blocking-output-of-kernel}

If you execute the next cell, you will see the output arriving as it is
generated, not all at the end.

    \begin{Verbatim}[commandchars=\\\{\}]
{\color{incolor}In [{\color{incolor}19}]:} \PY{k+kn}{import} \PY{n+nn}{time}\PY{o}{,} \PY{n+nn}{sys}
         \PY{k}{for} \PY{n}{i} \PY{o+ow}{in} \PY{n+nb}{range}\PY{p}{(}\PY{l+m+mi}{8}\PY{p}{)}\PY{p}{:}
             \PY{k}{print} \PY{n}{i}\PY{p}{,}
             \PY{n}{time}\PY{o}{.}\PY{n}{sleep}\PY{p}{(}\PY{l+m+mf}{0.5}\PY{p}{)}
\end{Verbatim}

    \begin{Verbatim}[commandchars=\\\{\}]
0 1 2 3 4 5 6 7
    \end{Verbatim}

    \subsection{Clean crash and restart}\label{clean-crash-and-restart}

We call the low-level system libc.time routine with the wrong argument
via ctypes to segfault the Python interpreter:

    \begin{Verbatim}[commandchars=\\\{\}]
{\color{incolor}In [{\color{incolor}*}]:} \PY{k+kn}{import} \PY{n+nn}{sys}
        \PY{k+kn}{from} \PY{n+nn}{ctypes} \PY{k+kn}{import} \PY{n}{CDLL}
        \PY{c}{\PYZsh{} This will crash a Linux or Mac system; equivalent calls can be made on Windows}
        \PY{n}{dll} \PY{o}{=} \PY{l+s}{\PYZsq{}}\PY{l+s}{dylib}\PY{l+s}{\PYZsq{}} \PY{k}{if} \PY{n}{sys}\PY{o}{.}\PY{n}{platform} \PY{o}{==} \PY{l+s}{\PYZsq{}}\PY{l+s}{darwin}\PY{l+s}{\PYZsq{}} \PY{k}{else} \PY{l+s}{\PYZsq{}}\PY{l+s}{.so.6}\PY{l+s}{\PYZsq{}}
        \PY{n}{libc} \PY{o}{=} \PY{n}{CDLL}\PY{p}{(}\PY{l+s}{\PYZdq{}}\PY{l+s}{libc.}\PY{l+s+si}{\PYZpc{}s}\PY{l+s}{\PYZdq{}} \PY{o}{\PYZpc{}} \PY{n}{dll}\PY{p}{)} 
        \PY{n}{libc}\PY{o}{.}\PY{n}{time}\PY{p}{(}\PY{o}{\PYZhy{}}\PY{l+m+mi}{1}\PY{p}{)}  \PY{c}{\PYZsh{} BOOM!!}
\end{Verbatim}

    \subsection{Markdown cells can contain formatted text and
code}\label{markdown-cells-can-contain-formatted-text-and-code}

You can \emph{italicize}, \textbf{boldface}

\begin{itemize}
\itemsep1pt\parskip0pt\parsep0pt
\item
  build
\item
  lists
\end{itemize}

and embed code meant for illustration instead of execution in Python:

\begin{verbatim}
def f(x):
    """a docstring"""
    return x**2
\end{verbatim}

or other languages:

\begin{verbatim}
if (i=0; i<n; i++) {
  printf("hello %d\n", i);
  x += 4;
}
\end{verbatim}

    Courtesy of MathJax, you can include mathematical expressions both
inline: $e^{i\pi} + 1 = 0$ and displayed:

\[e^x=\sum_{i=0}^\infty \frac{1}{i!}x^i\]

    \subsection{Rich displays: include anyting a browser can
show}\label{rich-displays-include-anyting-a-browser-can-show}

Note that we have an actual protocol for this, see the
\texttt{display\_protocol} notebook for further details.

\subsubsection{Images}\label{images}

    \begin{Verbatim}[commandchars=\\\{\}]
{\color{incolor}In [{\color{incolor}1}]:} \PY{k+kn}{from} \PY{n+nn}{IPython.display} \PY{k+kn}{import} \PY{n}{Image}
        \PY{n}{Image}\PY{p}{(}\PY{n}{filename}\PY{o}{=}\PY{l+s}{\PYZsq{}}\PY{l+s}{../../source/\PYZus{}static/logo.png}\PY{l+s}{\PYZsq{}}\PY{p}{)}
\end{Verbatim}
\texttt{\color{outcolor}Out[{\color{outcolor}1}]:}
    
    \begin{center}
    \adjustimage{max size={0.9\linewidth}{0.9\paperheight}}{Appendix_00_Notebook_Tour_files/Appendix_00_Notebook_Tour_23_0.png}
    \end{center}
    { \hspace*{\fill} \\}
    

    An image can also be displayed from raw data or a url

    \begin{Verbatim}[commandchars=\\\{\}]
{\color{incolor}In [{\color{incolor}2}]:} \PY{n}{Image}\PY{p}{(}\PY{n}{url}\PY{o}{=}\PY{l+s}{\PYZsq{}}\PY{l+s}{http://python.org/images/python\PYZhy{}logo.gif}\PY{l+s}{\PYZsq{}}\PY{p}{)}
\end{Verbatim}

            \begin{Verbatim}[commandchars=\\\{\}]
{\color{outcolor}Out[{\color{outcolor}2}]:} <IPython.core.display.Image at 0x1060e7410>
\end{Verbatim}
        
    SVG images are also supported out of the box (since modern browsers do a
good job of rendering them):

    \begin{Verbatim}[commandchars=\\\{\}]
{\color{incolor}In [{\color{incolor}3}]:} \PY{k+kn}{from} \PY{n+nn}{IPython.display} \PY{k+kn}{import} \PY{n}{SVG}
        \PY{n}{SVG}\PY{p}{(}\PY{n}{filename}\PY{o}{=}\PY{l+s}{\PYZsq{}}\PY{l+s}{python\PYZhy{}logo.svg}\PY{l+s}{\PYZsq{}}\PY{p}{)}
\end{Verbatim}
\texttt{\color{outcolor}Out[{\color{outcolor}3}]:}
    
    \begin{center}
    \adjustimage{max size={0.9\linewidth}{0.9\paperheight}}{Appendix_00_Notebook_Tour_files/Appendix_00_Notebook_Tour_27_0.pdf}
    \end{center}
    { \hspace*{\fill} \\}
    

    \paragraph{Embedded vs Non-embedded
Images}\label{embedded-vs-non-embedded-images}

    As of IPython 0.13, images are embedded by default for compatibility
with QtConsole, and the ability to still be displayed offline.

Let's look at the differences:

    \begin{Verbatim}[commandchars=\\\{\}]
{\color{incolor}In [{\color{incolor}4}]:} \PY{c}{\PYZsh{} by default Image data are embedded}
        \PY{n}{Embed}      \PY{o}{=} \PY{n}{Image}\PY{p}{(}    \PY{l+s}{\PYZsq{}}\PY{l+s}{http://scienceview.berkeley.edu/view/images/newview.jpg}\PY{l+s}{\PYZsq{}}\PY{p}{)}
        
        \PY{c}{\PYZsh{} if kwarg `url` is given, the embedding is assumed to be false}
        \PY{n}{SoftLinked} \PY{o}{=} \PY{n}{Image}\PY{p}{(}\PY{n}{url}\PY{o}{=}\PY{l+s}{\PYZsq{}}\PY{l+s}{http://scienceview.berkeley.edu/view/images/newview.jpg}\PY{l+s}{\PYZsq{}}\PY{p}{)}
        
        \PY{c}{\PYZsh{} In each case, embed can be specified explicitly with the `embed` kwarg}
        \PY{c}{\PYZsh{} ForceEmbed = Image(url=\PYZsq{}http://scienceview.berkeley.edu/view/images/newview.jpg\PYZsq{}, embed=True)}
\end{Verbatim}

    Today's image from a webcam at Berkeley, (at the time I created this
notebook). This should also work in the Qtconsole. Drawback is that the
saved notebook will be larger, but the image will still be present
offline.

    \begin{Verbatim}[commandchars=\\\{\}]
{\color{incolor}In [{\color{incolor}5}]:} \PY{n}{Embed}
\end{Verbatim}
\texttt{\color{outcolor}Out[{\color{outcolor}5}]:}
    
    \begin{center}
    \adjustimage{max size={0.9\linewidth}{0.9\paperheight}}{Appendix_00_Notebook_Tour_files/Appendix_00_Notebook_Tour_32_0.jpeg}
    \end{center}
    { \hspace*{\fill} \\}
    

    Today's image from same webcam at Berkeley, (refreshed every minutes, if
you reload the notebook), visible only with an active internet
connexion, that should be different from the previous one. This will not
work on Qtconsole. Notebook saved with this kind of image will be
lighter and always reflect the current version of the source, but the
image won't display offline.

    \begin{Verbatim}[commandchars=\\\{\}]
{\color{incolor}In [{\color{incolor}6}]:} \PY{n}{SoftLinked}
\end{Verbatim}

            \begin{Verbatim}[commandchars=\\\{\}]
{\color{outcolor}Out[{\color{outcolor}6}]:} <IPython.core.display.Image at 0x10fb99b10>
\end{Verbatim}
        
    Of course, if you re-run the all notebook, the two images will be the
same again.

    \subsubsection{Video}\label{video}

    And more exotic objects can also be displayed, as long as their
representation supports the IPython display protocol.

For example, videos hosted externally on YouTube are easy to load (and
writing a similar wrapper for other hosted content is trivial):

    \begin{Verbatim}[commandchars=\\\{\}]
{\color{incolor}In [{\color{incolor}7}]:} \PY{k+kn}{from} \PY{n+nn}{IPython.display} \PY{k+kn}{import} \PY{n}{YouTubeVideo}
        \PY{c}{\PYZsh{} a talk about IPython at Sage Days at U. Washington, Seattle.}
        \PY{c}{\PYZsh{} Video credit: William Stein.}
        \PY{n}{YouTubeVideo}\PY{p}{(}\PY{l+s}{\PYZsq{}}\PY{l+s}{1j\PYZus{}HxD4iLn8}\PY{l+s}{\PYZsq{}}\PY{p}{)}
\end{Verbatim}

            \begin{Verbatim}[commandchars=\\\{\}]
{\color{outcolor}Out[{\color{outcolor}7}]:} <IPython.lib.display.YouTubeVideo at 0x10fba2190>
\end{Verbatim}
        
    Using the nascent video capabilities of modern browsers, you may also be
able to display local videos. At the moment this doesn't work very well
in all browsers, so it may or may not work for you; we will continue
testing this and looking for ways to make it more robust.

The following cell loads a local file called \texttt{animation.m4v},
encodes the raw video as base64 for http transport, and uses the HTML5
video tag to load it. On Chrome 15 it works correctly, displaying a
control bar at the bottom with a play/pause button and a location
slider.

    \begin{Verbatim}[commandchars=\\\{\}]
{\color{incolor}In [{\color{incolor}8}]:} \PY{k+kn}{from} \PY{n+nn}{IPython.display} \PY{k+kn}{import} \PY{n}{HTML}
        \PY{n}{video} \PY{o}{=} \PY{n+nb}{open}\PY{p}{(}\PY{l+s}{\PYZdq{}}\PY{l+s}{animation.m4v}\PY{l+s}{\PYZdq{}}\PY{p}{,} \PY{l+s}{\PYZdq{}}\PY{l+s}{rb}\PY{l+s}{\PYZdq{}}\PY{p}{)}\PY{o}{.}\PY{n}{read}\PY{p}{(}\PY{p}{)}
        \PY{n}{video\PYZus{}encoded} \PY{o}{=} \PY{n}{video}\PY{o}{.}\PY{n}{encode}\PY{p}{(}\PY{l+s}{\PYZdq{}}\PY{l+s}{base64}\PY{l+s}{\PYZdq{}}\PY{p}{)}
        \PY{n}{video\PYZus{}tag} \PY{o}{=} \PY{l+s}{\PYZsq{}}\PY{l+s}{\PYZlt{}video controls alt=}\PY{l+s}{\PYZdq{}}\PY{l+s}{test}\PY{l+s}{\PYZdq{}}\PY{l+s}{ src=}\PY{l+s}{\PYZdq{}}\PY{l+s}{data:video/x\PYZhy{}m4v;base64,\PYZob{}0\PYZcb{}}\PY{l+s}{\PYZdq{}}\PY{l+s}{\PYZgt{}}\PY{l+s}{\PYZsq{}}\PY{o}{.}\PY{n}{format}\PY{p}{(}\PY{n}{video\PYZus{}encoded}\PY{p}{)}
        \PY{n}{HTML}\PY{p}{(}\PY{n}{data}\PY{o}{=}\PY{n}{video\PYZus{}tag}\PY{p}{)}
\end{Verbatim}

            \begin{Verbatim}[commandchars=\\\{\}]
{\color{outcolor}Out[{\color{outcolor}8}]:} <IPython.core.display.HTML at 0x10fba28d0>
\end{Verbatim}
        
    \subsection{Local Files}\label{local-files}

The above examples embed images and video from the notebook filesystem
in the output areas of code cells. It is also possible to request these
files directly in markdown cells if they reside in the notebook
directory via relative urls prefixed with \texttt{files/}:

\begin{verbatim}
files/[subdirectory/]<filename>
\end{verbatim}

For example, in the example notebook folder, we have the Python logo,
addressed as:

\begin{verbatim}
<img src="python-logo.svg" />
\end{verbatim}

and a video with the HTML5 video tag:

\begin{verbatim}
<video controls src="animation.m4v" />
\end{verbatim}

These do not embed the data into the notebook file, and require that the
files exist when you are viewing the notebook.

\subsubsection{Security of local files}\label{security-of-local-files}

Note that this means that the IPython notebook server also acts as a
generic file server for files inside the same tree as your notebooks.
Access is not granted outside the notebook folder so you have strict
control over what files are visible, but for this reason it is highly
recommended that you do not run the notebook server with a notebook
directory at a high level in your filesystem (e.g.~your home directory).

When you run the notebook in a password-protected manner, local file
access is restricted to authenticated users unless read-only views are
active.

    \subsubsection{External sites}\label{external-sites}

You can even embed an entire page from another site in an iframe; for
example this is today's Wikipedia page for mobile users:

    \begin{Verbatim}[commandchars=\\\{\}]
{\color{incolor}In [{\color{incolor}9}]:} \PY{n}{HTML}\PY{p}{(}\PY{l+s}{\PYZsq{}}\PY{l+s}{\PYZlt{}iframe src=http://en.mobile.wikipedia.org/?useformat=mobile width=700 height=350\PYZgt{}\PYZlt{}/iframe\PYZgt{}}\PY{l+s}{\PYZsq{}}\PY{p}{)}
\end{Verbatim}

            \begin{Verbatim}[commandchars=\\\{\}]
{\color{outcolor}Out[{\color{outcolor}9}]:} <IPython.core.display.HTML at 0x1094900d0>
\end{Verbatim}
        
    \subsubsection{Mathematics}\label{mathematics}

And we also support the display of mathematical expressions typeset in
LaTeX, which is rendered in the browser thanks to the
\href{http://mathjax.org}{MathJax library}.

Note that this is \emph{different} from the above examples. Above we
were typing mathematical expressions in Markdown cells (along with
normal text) and letting the browser render them; now we are displaying
the output of a Python computation as a LaTeX expression wrapped by the
\texttt{Math()} object so the browser renders it. The \texttt{Math}
object will add the needed LaTeX delimiters (\texttt{\$\$}) if they are
not provided:

    \begin{Verbatim}[commandchars=\\\{\}]
{\color{incolor}In [{\color{incolor}10}]:} \PY{k+kn}{from} \PY{n+nn}{IPython.display} \PY{k+kn}{import} \PY{n}{Math}
         \PY{n}{Math}\PY{p}{(}\PY{l+s}{r\PYZsq{}}\PY{l+s}{F(k) = }\PY{l+s}{\PYZbs{}}\PY{l+s}{int\PYZus{}\PYZob{}\PYZhy{}}\PY{l+s}{\PYZbs{}}\PY{l+s}{infty\PYZcb{}\PYZca{}\PYZob{}}\PY{l+s}{\PYZbs{}}\PY{l+s}{infty\PYZcb{} f(x) e\PYZca{}\PYZob{}2}\PY{l+s}{\PYZbs{}}\PY{l+s}{pi i k\PYZcb{} dx}\PY{l+s}{\PYZsq{}}\PY{p}{)}
\end{Verbatim}
\texttt{\color{outcolor}Out[{\color{outcolor}10}]:}
    
    
        \begin{equation*}
        F(k) = \int_{-\infty}^{\infty} f(x) e^{2\pi i k} dx
        \end{equation*}

    

    With the \texttt{Latex} class, you have to include the delimiters
yourself. This allows you to use other LaTeX modes such as
\texttt{eqnarray}:

    \begin{Verbatim}[commandchars=\\\{\}]
{\color{incolor}In [{\color{incolor}11}]:} \PY{k+kn}{from} \PY{n+nn}{IPython.display} \PY{k+kn}{import} \PY{n}{Latex}
         \PY{n}{Latex}\PY{p}{(}\PY{l+s}{r\PYZdq{}\PYZdq{}\PYZdq{}}\PY{l+s}{\PYZbs{}}\PY{l+s}{begin\PYZob{}eqnarray\PYZcb{}}
         \PY{l+s}{\PYZbs{}}\PY{l+s}{nabla }\PY{l+s}{\PYZbs{}}\PY{l+s}{times }\PY{l+s}{\PYZbs{}}\PY{l+s}{vec\PYZob{}}\PY{l+s}{\PYZbs{}}\PY{l+s}{mathbf\PYZob{}B\PYZcb{}\PYZcb{} \PYZhy{}}\PY{l+s}{\PYZbs{}}\PY{l+s}{, }\PY{l+s}{\PYZbs{}}\PY{l+s}{frac1c}\PY{l+s}{\PYZbs{}}\PY{l+s}{, }\PY{l+s}{\PYZbs{}}\PY{l+s}{frac\PYZob{}}\PY{l+s}{\PYZbs{}}\PY{l+s}{partial}\PY{l+s}{\PYZbs{}}\PY{l+s}{vec\PYZob{}}\PY{l+s}{\PYZbs{}}\PY{l+s}{mathbf\PYZob{}E\PYZcb{}\PYZcb{}\PYZcb{}\PYZob{}}\PY{l+s}{\PYZbs{}}\PY{l+s}{partial t\PYZcb{} \PYZam{} = }\PY{l+s}{\PYZbs{}}\PY{l+s}{frac\PYZob{}4}\PY{l+s}{\PYZbs{}}\PY{l+s}{pi\PYZcb{}\PYZob{}c\PYZcb{}}\PY{l+s}{\PYZbs{}}\PY{l+s}{vec\PYZob{}}\PY{l+s}{\PYZbs{}}\PY{l+s}{mathbf\PYZob{}j\PYZcb{}\PYZcb{} }\PY{l+s}{\PYZbs{}}\PY{l+s}{\PYZbs{}}
         \PY{l+s}{\PYZbs{}}\PY{l+s}{nabla }\PY{l+s}{\PYZbs{}}\PY{l+s}{cdot }\PY{l+s}{\PYZbs{}}\PY{l+s}{vec\PYZob{}}\PY{l+s}{\PYZbs{}}\PY{l+s}{mathbf\PYZob{}E\PYZcb{}\PYZcb{} \PYZam{} = 4 }\PY{l+s}{\PYZbs{}}\PY{l+s}{pi }\PY{l+s}{\PYZbs{}}\PY{l+s}{rho }\PY{l+s}{\PYZbs{}}\PY{l+s}{\PYZbs{}}
         \PY{l+s}{\PYZbs{}}\PY{l+s}{nabla }\PY{l+s}{\PYZbs{}}\PY{l+s}{times }\PY{l+s}{\PYZbs{}}\PY{l+s}{vec\PYZob{}}\PY{l+s}{\PYZbs{}}\PY{l+s}{mathbf\PYZob{}E\PYZcb{}\PYZcb{}}\PY{l+s}{\PYZbs{}}\PY{l+s}{, +}\PY{l+s}{\PYZbs{}}\PY{l+s}{, }\PY{l+s}{\PYZbs{}}\PY{l+s}{frac1c}\PY{l+s}{\PYZbs{}}\PY{l+s}{, }\PY{l+s}{\PYZbs{}}\PY{l+s}{frac\PYZob{}}\PY{l+s}{\PYZbs{}}\PY{l+s}{partial}\PY{l+s}{\PYZbs{}}\PY{l+s}{vec\PYZob{}}\PY{l+s}{\PYZbs{}}\PY{l+s}{mathbf\PYZob{}B\PYZcb{}\PYZcb{}\PYZcb{}\PYZob{}}\PY{l+s}{\PYZbs{}}\PY{l+s}{partial t\PYZcb{} \PYZam{} = }\PY{l+s}{\PYZbs{}}\PY{l+s}{vec\PYZob{}}\PY{l+s}{\PYZbs{}}\PY{l+s}{mathbf\PYZob{}0\PYZcb{}\PYZcb{} }\PY{l+s}{\PYZbs{}}\PY{l+s}{\PYZbs{}}
         \PY{l+s}{\PYZbs{}}\PY{l+s}{nabla }\PY{l+s}{\PYZbs{}}\PY{l+s}{cdot }\PY{l+s}{\PYZbs{}}\PY{l+s}{vec\PYZob{}}\PY{l+s}{\PYZbs{}}\PY{l+s}{mathbf\PYZob{}B\PYZcb{}\PYZcb{} \PYZam{} = 0 }
         \PY{l+s}{\PYZbs{}}\PY{l+s}{end\PYZob{}eqnarray\PYZcb{}}\PY{l+s}{\PYZdq{}\PYZdq{}\PYZdq{}}\PY{p}{)}
\end{Verbatim}
\texttt{\color{outcolor}Out[{\color{outcolor}11}]:}
    
    \begin{eqnarray}
\nabla \times \vec{\mathbf{B}} -\, \frac1c\, \frac{\partial\vec{\mathbf{E}}}{\partial t} & = \frac{4\pi}{c}\vec{\mathbf{j}} \\
\nabla \cdot \vec{\mathbf{E}} & = 4 \pi \rho \\
\nabla \times \vec{\mathbf{E}}\, +\, \frac1c\, \frac{\partial\vec{\mathbf{B}}}{\partial t} & = \vec{\mathbf{0}} \\
\nabla \cdot \vec{\mathbf{B}} & = 0 
\end{eqnarray}

    

    Or you can enter latex directly with the \texttt{\%\%latex} cell magic:

    \begin{Verbatim}[commandchars=\\\{\}]
{\color{incolor}In [{\color{incolor}12}]:} \PY{o}{\PYZpc{}\PYZpc{}}\PY{k}{latex}
         \PYZbs{}\PY{n}{begin}\PY{p}{\PYZob{}}\PY{n}{aligned}\PY{p}{\PYZcb{}}
         \PYZbs{}\PY{n}{nabla} \PYZbs{}\PY{n}{times} \PYZbs{}\PY{n}{vec}\PY{p}{\PYZob{}}\PYZbs{}\PY{n}{mathbf}\PY{p}{\PYZob{}}\PY{n}{B}\PY{p}{\PYZcb{}}\PY{p}{\PYZcb{}} \PY{o}{\PYZhy{}}\PYZbs{}\PY{p}{,} \PYZbs{}\PY{n}{frac1c}\PYZbs{}\PY{p}{,} \PYZbs{}\PY{n}{frac}\PY{p}{\PYZob{}}\PYZbs{}\PY{n}{partial}\PYZbs{}\PY{n}{vec}\PY{p}{\PYZob{}}\PYZbs{}\PY{n}{mathbf}\PY{p}{\PYZob{}}\PY{n}{E}\PY{p}{\PYZcb{}}\PY{p}{\PYZcb{}}\PY{p}{\PYZcb{}}\PY{p}{\PYZob{}}\PYZbs{}\PY{n}{partial} \PY{n}{t}\PY{p}{\PYZcb{}} \PY{o}{\PYZam{}} \PY{o}{=} \PYZbs{}\PY{n}{frac}\PY{p}{\PYZob{}}\PY{l+m+mi}{4}\PYZbs{}\PY{n}{pi}\PY{p}{\PYZcb{}}\PY{p}{\PYZob{}}\PY{n}{c}\PY{p}{\PYZcb{}}\PYZbs{}\PY{n}{vec}\PY{p}{\PYZob{}}\PYZbs{}\PY{n}{mathbf}\PY{p}{\PYZob{}}\PY{n}{j}\PY{p}{\PYZcb{}}\PY{p}{\PYZcb{}} \PYZbs{}\PYZbs{}
         \PYZbs{}\PY{n}{nabla} \PYZbs{}\PY{n}{cdot} \PYZbs{}\PY{n}{vec}\PY{p}{\PYZob{}}\PYZbs{}\PY{n}{mathbf}\PY{p}{\PYZob{}}\PY{n}{E}\PY{p}{\PYZcb{}}\PY{p}{\PYZcb{}} \PY{o}{\PYZam{}} \PY{o}{=} \PY{l+m+mi}{4} \PYZbs{}\PY{n}{pi} \PYZbs{}\PY{n}{rho} \PYZbs{}\PYZbs{}
         \PYZbs{}\PY{n}{nabla} \PYZbs{}\PY{n}{times} \PYZbs{}\PY{n}{vec}\PY{p}{\PYZob{}}\PYZbs{}\PY{n}{mathbf}\PY{p}{\PYZob{}}\PY{n}{E}\PY{p}{\PYZcb{}}\PY{p}{\PYZcb{}}\PYZbs{}\PY{p}{,} \PY{o}{+}\PYZbs{}\PY{p}{,} \PYZbs{}\PY{n}{frac1c}\PYZbs{}\PY{p}{,} \PYZbs{}\PY{n}{frac}\PY{p}{\PYZob{}}\PYZbs{}\PY{n}{partial}\PYZbs{}\PY{n}{vec}\PY{p}{\PYZob{}}\PYZbs{}\PY{n}{mathbf}\PY{p}{\PYZob{}}\PY{n}{B}\PY{p}{\PYZcb{}}\PY{p}{\PYZcb{}}\PY{p}{\PYZcb{}}\PY{p}{\PYZob{}}\PYZbs{}\PY{n}{partial} \PY{n}{t}\PY{p}{\PYZcb{}} \PY{o}{\PYZam{}} \PY{o}{=} \PYZbs{}\PY{n}{vec}\PY{p}{\PYZob{}}\PYZbs{}\PY{n}{mathbf}\PY{p}{\PYZob{}}\PY{l+m+mi}{0}\PY{p}{\PYZcb{}}\PY{p}{\PYZcb{}} \PYZbs{}\PYZbs{}
         \PYZbs{}\PY{n}{nabla} \PYZbs{}\PY{n}{cdot} \PYZbs{}\PY{n}{vec}\PY{p}{\PYZob{}}\PYZbs{}\PY{n}{mathbf}\PY{p}{\PYZob{}}\PY{n}{B}\PY{p}{\PYZcb{}}\PY{p}{\PYZcb{}} \PY{o}{\PYZam{}} \PY{o}{=} \PY{l+m+mi}{0}
         \PYZbs{}\PY{n}{end}\PY{p}{\PYZob{}}\PY{n}{aligned}\PY{p}{\PYZcb{}}
\end{Verbatim}

    \begin{aligned}
\nabla \times \vec{\mathbf{B}} -\, \frac1c\, \frac{\partial\vec{\mathbf{E}}}{\partial t} & = \frac{4\pi}{c}\vec{\mathbf{j}} \\
\nabla \cdot \vec{\mathbf{E}} & = 4 \pi \rho \\
\nabla \times \vec{\mathbf{E}}\, +\, \frac1c\, \frac{\partial\vec{\mathbf{B}}}{\partial t} & = \vec{\mathbf{0}} \\
\nabla \cdot \vec{\mathbf{B}} & = 0
\end{aligned}

    
    There is also a \texttt{\%\%javascript} cell magic for running
javascript directly, and \texttt{\%\%svg} for manually entering SVG
content.

    \section{Loading external codes}\label{loading-external-codes}

\begin{itemize}
\itemsep1pt\parskip0pt\parsep0pt
\item
  Drag and drop a \texttt{.py} in the dashboard
\item
  Use \texttt{\%load} with any local or remote url:
  \href{http://matplotlib.sourceforge.net/gallery.html}{the Matplotlib
  Gallery!}
\end{itemize}

In this notebook we've kept the output saved so you can see the result,
but you should run the next cell yourself (with an active internet
connection).

    Let's make sure we have pylab again, in case we have restarted the
kernel due to the crash demo above

    \begin{Verbatim}[commandchars=\\\{\}]
{\color{incolor}In [{\color{incolor}12}]:} \PY{o}{\PYZpc{}}\PY{k}{pylab} \PY{n}{inline}
\end{Verbatim}

    \begin{Verbatim}[commandchars=\\\{\}]
Welcome to pylab, a matplotlib-based Python environment [backend: module://IPython.zmq.pylab.backend\_inline].
For more information, type 'help(pylab)'.
    \end{Verbatim}

    \begin{Verbatim}[commandchars=\\\{\}]
{\color{incolor}In [{\color{incolor}15}]:} \PY{o}{\PYZpc{}}\PY{k}{load} \PY{n}{http}\PY{p}{:}\PY{o}{/}\PY{o}{/}\PY{n}{matplotlib}\PY{o}{.}\PY{n}{sourceforge}\PY{o}{.}\PY{n}{net}\PY{o}{/}\PY{n}{mpl\PYZus{}examples}\PY{o}{/}\PY{n}{pylab\PYZus{}examples}\PY{o}{/}\PY{n}{integral\PYZus{}demo}\PY{o}{.}\PY{n}{py}
\end{Verbatim}

    \begin{Verbatim}[commandchars=\\\{\}]
{\color{incolor}In [{\color{incolor}16}]:} \PY{c}{\PYZsh{}!/usr/bin/env python}
         
         \PY{c}{\PYZsh{} implement the example graphs/integral from pyx}
         \PY{k+kn}{from} \PY{n+nn}{pylab} \PY{k+kn}{import} \PY{o}{*}
         \PY{k+kn}{from} \PY{n+nn}{matplotlib.patches} \PY{k+kn}{import} \PY{n}{Polygon}
         
         \PY{k}{def} \PY{n+nf}{func}\PY{p}{(}\PY{n}{x}\PY{p}{)}\PY{p}{:}
             \PY{k}{return} \PY{p}{(}\PY{n}{x}\PY{o}{\PYZhy{}}\PY{l+m+mi}{3}\PY{p}{)}\PY{o}{*}\PY{p}{(}\PY{n}{x}\PY{o}{\PYZhy{}}\PY{l+m+mi}{5}\PY{p}{)}\PY{o}{*}\PY{p}{(}\PY{n}{x}\PY{o}{\PYZhy{}}\PY{l+m+mi}{7}\PY{p}{)}\PY{o}{+}\PY{l+m+mi}{85}
         
         \PY{n}{ax} \PY{o}{=} \PY{n}{subplot}\PY{p}{(}\PY{l+m+mi}{111}\PY{p}{)}
         
         \PY{n}{a}\PY{p}{,} \PY{n}{b} \PY{o}{=} \PY{l+m+mi}{2}\PY{p}{,} \PY{l+m+mi}{9} \PY{c}{\PYZsh{} integral area}
         \PY{n}{x} \PY{o}{=} \PY{n}{arange}\PY{p}{(}\PY{l+m+mi}{0}\PY{p}{,} \PY{l+m+mi}{10}\PY{p}{,} \PY{l+m+mf}{0.01}\PY{p}{)}
         \PY{n}{y} \PY{o}{=} \PY{n}{func}\PY{p}{(}\PY{n}{x}\PY{p}{)}
         \PY{n}{plot}\PY{p}{(}\PY{n}{x}\PY{p}{,} \PY{n}{y}\PY{p}{,} \PY{n}{linewidth}\PY{o}{=}\PY{l+m+mi}{1}\PY{p}{)}
         
         \PY{c}{\PYZsh{} make the shaded region}
         \PY{n}{ix} \PY{o}{=} \PY{n}{arange}\PY{p}{(}\PY{n}{a}\PY{p}{,} \PY{n}{b}\PY{p}{,} \PY{l+m+mf}{0.01}\PY{p}{)}
         \PY{n}{iy} \PY{o}{=} \PY{n}{func}\PY{p}{(}\PY{n}{ix}\PY{p}{)}
         \PY{n}{verts} \PY{o}{=} \PY{p}{[}\PY{p}{(}\PY{n}{a}\PY{p}{,}\PY{l+m+mi}{0}\PY{p}{)}\PY{p}{]} \PY{o}{+} \PY{n+nb}{zip}\PY{p}{(}\PY{n}{ix}\PY{p}{,}\PY{n}{iy}\PY{p}{)} \PY{o}{+} \PY{p}{[}\PY{p}{(}\PY{n}{b}\PY{p}{,}\PY{l+m+mi}{0}\PY{p}{)}\PY{p}{]}
         \PY{n}{poly} \PY{o}{=} \PY{n}{Polygon}\PY{p}{(}\PY{n}{verts}\PY{p}{,} \PY{n}{facecolor}\PY{o}{=}\PY{l+s}{\PYZsq{}}\PY{l+s}{0.8}\PY{l+s}{\PYZsq{}}\PY{p}{,} \PY{n}{edgecolor}\PY{o}{=}\PY{l+s}{\PYZsq{}}\PY{l+s}{k}\PY{l+s}{\PYZsq{}}\PY{p}{)}
         \PY{n}{ax}\PY{o}{.}\PY{n}{add\PYZus{}patch}\PY{p}{(}\PY{n}{poly}\PY{p}{)}
         
         \PY{n}{text}\PY{p}{(}\PY{l+m+mf}{0.5} \PY{o}{*} \PY{p}{(}\PY{n}{a} \PY{o}{+} \PY{n}{b}\PY{p}{)}\PY{p}{,} \PY{l+m+mi}{30}\PY{p}{,}
              \PY{l+s}{r\PYZdq{}}\PY{l+s}{\PYZdl{}}\PY{l+s}{\PYZbs{}}\PY{l+s}{int\PYZus{}a\PYZca{}b f(x)}\PY{l+s}{\PYZbs{}}\PY{l+s}{mathrm\PYZob{}d\PYZcb{}x\PYZdl{}}\PY{l+s}{\PYZdq{}}\PY{p}{,} \PY{n}{horizontalalignment}\PY{o}{=}\PY{l+s}{\PYZsq{}}\PY{l+s}{center}\PY{l+s}{\PYZsq{}}\PY{p}{,}
              \PY{n}{fontsize}\PY{o}{=}\PY{l+m+mi}{20}\PY{p}{)}
         
         \PY{n}{axis}\PY{p}{(}\PY{p}{[}\PY{l+m+mi}{0}\PY{p}{,}\PY{l+m+mi}{10}\PY{p}{,} \PY{l+m+mi}{0}\PY{p}{,} \PY{l+m+mi}{180}\PY{p}{]}\PY{p}{)}
         \PY{n}{figtext}\PY{p}{(}\PY{l+m+mf}{0.9}\PY{p}{,} \PY{l+m+mf}{0.05}\PY{p}{,} \PY{l+s}{\PYZsq{}}\PY{l+s}{x}\PY{l+s}{\PYZsq{}}\PY{p}{)}
         \PY{n}{figtext}\PY{p}{(}\PY{l+m+mf}{0.1}\PY{p}{,} \PY{l+m+mf}{0.9}\PY{p}{,} \PY{l+s}{\PYZsq{}}\PY{l+s}{y}\PY{l+s}{\PYZsq{}}\PY{p}{)}
         \PY{n}{ax}\PY{o}{.}\PY{n}{set\PYZus{}xticks}\PY{p}{(}\PY{p}{(}\PY{n}{a}\PY{p}{,}\PY{n}{b}\PY{p}{)}\PY{p}{)}
         \PY{n}{ax}\PY{o}{.}\PY{n}{set\PYZus{}xticklabels}\PY{p}{(}\PY{p}{(}\PY{l+s}{\PYZsq{}}\PY{l+s}{a}\PY{l+s}{\PYZsq{}}\PY{p}{,}\PY{l+s}{\PYZsq{}}\PY{l+s}{b}\PY{l+s}{\PYZsq{}}\PY{p}{)}\PY{p}{)}
         \PY{n}{ax}\PY{o}{.}\PY{n}{set\PYZus{}yticks}\PY{p}{(}\PY{p}{[}\PY{p}{]}\PY{p}{)}
         \PY{n}{show}\PY{p}{(}\PY{p}{)}
\end{Verbatim}

    \begin{center}
    \adjustimage{max size={0.9\linewidth}{0.9\paperheight}}{Appendix_00_Notebook_Tour_files/Appendix_00_Notebook_Tour_55_0.png}
    \end{center}
    { \hspace*{\fill} \\}
    

    % Add a bibliography block to the postdoc
    
    
    
    \end{document}
